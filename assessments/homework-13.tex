\documentclass{article}
\usepackage{style-assessments}

% define macros (/shortcuts)
\newcommand{\blankul}[1]{\rule[-1.5mm]{#1}{0.15mm}}	% shortcut for blank underline, where the only option needed to specify is length (# and units (cm or mm, etc.)))
\newcommand{\follow}[1]{\sim \text{#1}\,}		% shortcut for ~ 'Named dist ' in normal font with space before parameters would go
\newcommand{\e}{\mathrm{e}}		% shortcut for non-italic e in math mode

\begin{document}

\hspace{375pt}Name:

\begin{center}
{\Huge MATH 320: Homework 13}
\end{center}

\bigskip\bigskip

{\large \textbf{Due \blankul{4cm}}: Turn in a hard copy, neat and stapled.}\bigskip

% problem types summary
% 1) Expected value with a) deductible and b) cap
% 2) Working backwards expected value of deductible
% 3) Expected value of a function of a random variable (works like a cap)
% 3) Cdf pdf and probability of transformation (decreasing function)
% 4) Cdf and probability of a transformation (increasing function)

\begin{enumerate}
    \item Assume the amount of a single loss for an insurance policy has the density function $f(x) = 0.05 \e^{-0.05x}$, for $x > 0$.%Actex 9-1 and 9-2 (with different numbers)
    \begin{enumerate}
        \item Suppose this policy has a \$5 per claim deductible. Find the expected amount of a single claim for this policy.
        \item Now suppose there is a payment cap of \$30 (and no deductible). Find the expected amount of a single claim for this policy.
    \end{enumerate}\bigskip
    
    \item An insurance policy is written to cover a loss, $X$, where $X \follow{Uniform}(a = 0, b = 1000)$.
    \item[] At what level must a deductible be set in order for the expected payment to be 25\% of what it would be with no deductible?\bigskip% Actex 9-28
    
    \item A device that continuously measures and records seismic activity is placed in a remote region. The time, $T$, to failure of this device is uniformly distributed on the interval $[0,80]$ years.
    \item[] Since the device will not be monitored during its first 10 years of service, the time to discovery of its failure is $X = \text{max}(T, 10)$ (i.e. $X$ takes the value of whichever is greater for that particular $x$ point).% Acted 9-30 (different distribution)
    \begin{enumerate}
        \item Write $X$ as a piecewise function of $T$.%Original addition
        \item Find $E(X)$.%Actex 9-30 (different distribution)
    \end{enumerate}\bigskip
    
    \item Let $X \follow{Exponential}(\lambda = 0.5)$ and $Y = 1 / X$. Assume $x > 0$ for this problem.%Original
    \begin{enumerate}
        \item Find $F_Y(y)$.
        \item Find $f_Y(y)$.
        \item Find $P(1 \le Y \le 2)$.
    \end{enumerate}\bigskip
    
    \item An investment account earns an annual interest rate $R$ that follows a uniform distribution on the interval $(0.05, 0.08)$. The value of a 10,000 initial investment in this account after one year is given by $V = 10,000\e^R$.
    \begin{enumerate}
        \item Find $F_V(v)$.
        \item Find $P(V \ge 10,700)$.
    \end{enumerate}\bigskip
  
\end{enumerate}

\newpage

Select answers\bigskip
\begin{enumerate}
    \item 
    \begin{enumerate}
        \item $\text{Exp value} = 15.576$
        \item $\text{Exp value} = 15.537$
    \end{enumerate}
    
    \item $d = 500$
    
    \item 
    \begin{enumerate}
        \item 
        \item $E(X) = 40.625$
    \end{enumerate}
      
    \item
    \begin{enumerate}
        \item 
        \item 
        \item $\text{Prob} \approx 0.1723$
    \end{enumerate}
    
    \item 
    \begin{enumerate}
        \item 
        \item $\text{Prob} \approx 0.5886$
    \end{enumerate}
    
\end{enumerate}
    
\end{document}