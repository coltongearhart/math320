\documentclass{article}
\usepackage{style-assessments}

% define macros (/shortcuts)
\newcommand{\blankul}[1]{\rule[-1.5mm]{#1}{0.15mm}}	% shortcut for blank underline, where the only option needed to specify is length (# and units (cm or mm, etc.)))
\newcommand{\comp}{{\sim}}						% shortcut for tilde without extra space, using for complement

\begin{document}

\hspace{375pt}Name:

\begin{center}
{\Huge MATH 320: Homework 3}
\end{center}

\bigskip\bigskip

{\large \textbf{Due \blankul{4cm}}: Turn in a hard copy, neat and stapled.}\bigskip

% problem types summary
% 1) Combinations to find probabilities
% 2) Permutations to find probabilities
% 3) Venn Diagram probability
% 4) Proof of inequality
% 5) Venn Diagram probability and set theory / probability theorem probabilities
% 6) Combinations to find probabilities (harder)

% next semester, add problems from different sources (drew's or Math stats with apps)

\begin{enumerate}
    \item A computer company has a shipment of 60 computer components of which 12 are defective. If 8 components are chosen at random to be tested, what is the probability that:
    \begin{enumerate}
        \item All are good?
        \item 5 are good and 3 are defective?
        \item At least 6 are defective?% Original addition
    \end{enumerate}\bigskip%Actex 3-8 (with different numbers)
    
    \item 12 people, 6 men and 6 women, are to be seated in a row of 12 chairs. What is the probability that the men and women end up in alternate chairs?\bigskip%Actex 3-9 (with different numbers)
    
    \item An auto insurance company finds in the past 10 years 25\% of its policyholders have filed liability claims, 33\% have filed comprehensive claims, and 18\% have filed both types of claims. What is the probability that a policyholder chosen at random has not filed a claim of either kind?\bigskip%Actex 3-20 (with different numbers)
    
    \item Prove $P(A \cap B) \ge P(A) + P(B) - 1.$%Theory notes, lecture 2
    \item[] Note that this theorem allows us to place a lower bound on the probability of simultaneous events (intersection) in terms of the probabilities of the individual events.\bigskip
    \item You are given $P(A \cup B) = 0.6$ and $P(A \cup \comp B) = 0.85.$ Determine $P(A)$ using each of the following methods:%Actex 3-47 (with different numbers)
    \begin{enumerate}
        \item Venn Diagram.
        \item Set theory and probability theorems.
    \end{enumerate}\bigskip
    
    \item If 5 cards are dealt from a deck of 52 ordinary playing cards, find the probability of:% Theory Lecture 3 example
    \begin{enumerate}
        \item A ``full house''. Note that a fill house contains three matching cards of one rank and two matching cards of another rank. \textit{HINT: Think of selecting the rank and the suit as separate tasks.}
        \item A hand of one pair. Note that one pair contains two cards of the same rank and three cards of three other ranks.
    \end{enumerate} 
\end{enumerate}

\newpage

Select answers\bigskip
\begin{enumerate}
    \item 
    \begin{enumerate}
        \item $\text{Prob} \approx 0.147$
        \item $\text{Prob} \approx 0.147$
        \item $\text{Prob} \approx 0.809$
    \end{enumerate}
    
    \item $\text{Prob} \approx 0.0022$
    
    \item $\text{Prob} = 0.6$
    
    \item 
    
    \item $\text{Prob} = 0.45$
    
    \item 
    \begin{enumerate}
        \item $\text{Prob} \approx 0.0014$
        \item $\text{Prob} \approx 0.4226$
    \end{enumerate}
\end{enumerate}

\end{document}