\documentclass{article}
\usepackage{style-syllabus}

\begin{document}

\begin{center}
\Huge{MATH 320: Probability}

\large{Fall 2023, Ball State University}
\end{center}

\bigskip\bigskip

\textbf{\large Course Information} (MATH 320-1)\medskip

\begin{tabular}{ll}
    \textbf{Class}: & MTRF 11:00 - 11:50 AM, Robert Bell 115\\\\
    \textbf{Instructor}: & Colton Gearhart (email: colton.gearhart@bsu.edu) \\
     & You can expect a response to emails within 24 - 48 hours \\\\
    \textbf{Office Hours}: & Tentatively TR 12:00 - 1:00 PM or by appointment, Office RB 411 \\
     & \textbf{Please make appointment for office hours}
\end{tabular}\bigskip

\textbf{\large Course Materials}\medskip

\begin{tabular}{p{0.2\linewidth}p{0.75\linewidth}}
    \textbf{Textbook}: & Probability and Statistical Inference, 10th ed., Hogg, Tanis, and Zimmerman, Pearson 2020 (Recommended)\\\\
    \textbf{Calculator}: & TI-30XS MultiView or TI-84 Graphing Calculator\\\\
    \textbf{Course Website} & Assignments, solutions and some other class materials will be posted on Canvas
\end{tabular}\bigskip

\textbf{\large Course Description}\medskip

Probability theory for discrete and continuous sample spaces, random variables, density functions, distribution functions, marginal and conditional distributions, mathematical expectation, moment-generating functions, common distributions, sampling distribution theory, central limit theorem, t, chi-square, and F distributions.

Prerequisites: C- or better in MATH 166, MATH 215 (can be taken in parallel).\\
4 Credit hours (4 Lecture hours).\bigskip

\textbf{\large Course Objectives}\medskip

\begin{itemize}
    \item Use counting methods and probability rules to calculate probabilities, including conditional
    probabilities.
    \item Use univariate discrete and continuous distributions, including those from common parametric
    families of distributions, to calculate probabilities, expected values and variances.
    \item Derive and identify moment generating functions of parametric families of distributions; use moment generating functions to compute moments and to identify underlying probability
    distributions.
    \item Use bivariate discrete and continuous distributions to calculate probabilities, expected values,
    covariance and correlation.
    \item Use sampling distribution theory to find distributions of functions of random variables and sums
    of independent random variables, including when sampling from a normal distribution.
    \item Use the Central Limit Theorem to approximate distributions and calculate probabilities.
\end{itemize}\bigskip

\textbf{\large Course Rationale}\medskip

The concepts of probability are of great importance in a wide variety of applications.
The theory of probability, as the foundation upon which the methods of statistics are based, should command the attention of those who what to understand as well as apply statistical techniques. This course, therefore, is a required course for those who want to major in statistics or actuarial science and is an excellent course for those who are in mathematics, business, and other allied fields.\bigskip

\textbf{\large Course Content}\medskip

We will study chapters 1 -- 5 of the textbook (time permitting), which cover the following topics:
\begin{enumerate}
    \item Probability
    \item[] Random Experiments, Random Variables, Properties of Probability, Methods of Enumeration, Conditional Probability, Bayes' theorem, Independent Events.
    \item Distributions of Discrete Type
    \item[] Random Variables of Discrete Type, Mathematical Expectation, Mean and Variance, Moment generating functions, Bernoulli and Binomial Distributions, Geometric and Negative Binomial Distributions, Multivariate Distributions of Discrete Type, Correlation Coefficient, Conditional Distributions, Multinomial Distribution.
    \item Distributions of Continuous Type
    \item[] Samples, Histograms, and Ogives, Exploratory Data Analysis, Random Variables of continuous Type, Uniform Distribution, Exponential and Gamma Distributions, Normal Distribution, Multivariate Distributions of Continuous Type, Bivariate Normal Distribution, Sampling from Bivariate Distributions, Mixed Distributions and Censoring.
    \item Sampling Distribution Theory
    \item[] Distributions of functions of Random Variables, Sums of Independent Random Variables, Chi-square Distribution, The t and F Distributions, Central Limit theorem, Approximations for Discrete Distributions, Limiting Moment generating Functions, Transformations of Random Variables.
\end{enumerate}\bigskip

\textbf{\large In-Class Activities}\medskip

Throughout the semester, there will be in-class activities worth a total of 15\% of your grade. The lowest score of these activities will be dropped. The in-class tasks will be mainly designed to practice what you studied recently.

If you miss class when there is an in-class activity, you will receive a zero unless there is a legitimate reason pre-approved by me.\bigskip

\textbf{\large Homework}\medskip

Homework assignments will account for 20\% of the course grade.\medskip
\begin{itemize}
    \item The lowest homework assignment score will be dropped.
    \item Any assignment turned in within one week after the deadline may be given a 50\% grade reduction. After that, it will not be accepted. An exception to this reduction policy will be considered for legitimate circumstances that are presented to me (via e-mail or in person) \textit{before the due date}. 
    \item Submissions must be neat and stapled.
    \item Unorganized and/or illegible work will be considered incorrect.
\end{itemize}\medskip

\ul{Collaboration}: Feel free to work together, but the work you turn in should be your own. For example, if you solve a problem with a friend and the two of you write down identical solutions, you both will be in violation of this policy. Instead, after solving the problem together, you might re-solve it on your own so that it is evident the work is yours. If your work is not clearly independent of others', a score of 0 may be awarded for the entire offending assignment.\medskip

\ul{Requesting help}: The homework is essential to success in this class because it forms the foundation of knowledge and problem-solving skill upon which exam success can be built. I am glad to help you with the homework, but it is in your own best interest to work on a problem for awhile before you come for help, even if you are stuck on it.\medskip

\ul{Showing work}: In this class, correct answers are typically not enough to get full credit. Any work that you turn in (whether homework, in-class assignments, tests or final) must demonstrate the process by which the solution was obtained. This means that if you write down the correct answer with no supporting work you may receive partial credit or no credit at all.\bigskip

\textbf{\large Tests and Final}\medskip

There will be three Tests that together account for 45\% of the total grade (15\% per Test), each will be held during the entire class period.

At the end of the semester, there will be a comprehensive Final worth 20\% of the total grade. It will be held on Thursday, December 14 from 9:45 AM to 11:45 AM.\bigskip

\textbf{\large Grading}\medskip

Your final grade will be comprised of the following elements:\bigskip\\
\begin{tabular}{ll}
    \textbf{In-Class Assignments} & 15\%\\
    \textbf{Homework} & 20\%\\
    \textbf{Tests} & 45\% (3 Tests, each worth 15\%)\\
    \textbf{Final} & 20\% 
\end{tabular}\medskip

Letter grades:\bigskip\\
\begin{tabular}{ll | ll | ll | ll | ll}
    \hline
    $[93 - 100]$ & A & $[87 - 90)$ & B+ & $[77 - 80)$ & C+ & $[67 - 70)$ & D+ & $< 60$ & F\\
    $[90 - 93)$ & A-  & $[83 - 87)$ & B & $[73 - 77)$ & C & $[63 - 67)$ & D & \\
     & & $[80 - 83)$ & B- & $[70 - 73)$ & C- & $[60 - 63)$ & D- & \\
    \hline
\end{tabular}

I may lower the grade thresholds, but will not raise them.\bigskip

\textbf{\large Attendance Policy}\medskip

The pace of this class is such that it will not be advisable to miss any sessions. If you know you will be absent, please inform me in advance. When you are absent, it will be your responsibility to look on Canvas for the uploaded notes and announcements.

I really want to encourage you to ASK QUESTIONS and take part in the lectures! This is part of the learning process and benefits others in the class who may have the same questions.\bigskip

\newpage

\textbf{\large Important Dates}\medskip

\begin{tabular}{ll}
    August 21, Monday & Classes begin\\
    September 4, Monday & Labor Day, no class\\
    October 9 -- 10, Monday and Tuesday & Fall Break, no class\\
    October 30, Monday & \textbf{Last day to drop with "W"}\\
    November 22 -- 24, Wednesday -- Friday & Thanksgiving Break, no class\\
    December 11, Monday & Last day of class\\
    December 14, Thursday & Final Exam at 9:45 AM
\end{tabular}\bigskip

\textbf{\large Withdrawal Statement}\medskip

The course withdrawal period ends \textbf{Monday, October 30, 2023 at 5:00 PM}. Before this date, students can elect to receive a "W" for the course by completing and submitting the proper form. The instructor's permission is not required. For details, see \href{https://www.bsu.edu/about/administrativeoffices/registrar/registration-activities/withdraw-from-classes}{here} as well as Degree Requirements and Time Limits in the current Undergraduate Catalog OR Withdrawal Procedures in the current graduate catalog.\bigskip

\textbf{\large Disability Statement}\medskip

Do not hesitate to contact me with any questions or concerns. If you need course adaptations or accommodations because of a disability, please contact me as soon as possible. \href{https://www.bsu.edu/about/administrativeoffices/disability-services}{The Office of Disability Services} coordinates services for students with disabilities; documentation of a disability needs to be on file in that office before any accommodations can be provided. Disability Services can be contacted at 765-285-5293 or dsd@bsu.edu.

If you are experiencing mental health concerns, telehealth services are available ? here is a link to the \href{https://www.bsu.edu/campuslife/counseling-center}{Counseling Center website}.\bigskip

\textbf{\large Diversity Statement}\medskip

Ball State University aspires to be a university that attracts and retains a diverse faculty, staff, and student body. We are committed to ensuring that all members of the community are welcome, through valuing the various experiences and worldviews represented at Ball State and among those we serve. We promote a culture of respect and civil discourse as expressed in our \href{https://www.bsu.edu/about/beneficence}{Pledge Beneficence} and through university resources found \href{http://bsu.edu/campuslife/multiculturalcenter}{here}.\bigskip

\textbf{\large Important Links}\medskip

\href{https://www.bsu.edu/about/administrativeoffices/vice-provost/student-services/academic-integrity}{Student Academic Ethics Policy}

\href{https://www.bsu.edu/about/administrativeoffices/student-conduct/policiesandprocedures}{Code of Student Rights and Responsibilities}\bigskip

\center \textit{Syllabus is subject to change}

\end{document}